\chapter{UNARCHIVER}
\lstinline;UNARCHIVER; is the central piece for unarchiving. \lstinline;ARCHIVE;
will parse the header and search for the last registered (!)
\lstinline;UNARCHIVER; that can unarchive the payload that belongs to the
header. This \lstinline;UNARCHIVER; will then be initialized with the header
and be passed blocks until it indicates that unarchiving finished.

etar provides two predefined \lstinline;UNARCHIVER;s

\section{FILE\_UNARCHIVER}
\lstinline;FILE_UNARCHIVER; accepts all headers that have a typeflag for a
regular file ('0' and '\%U'). It will create a \lstinline;RAW_FILE; and copy all
payload blocks to it until \lstinline;size; bytes are written (indicated by the
header). Additionally it will try to set the metadata according to the header.

\section{DIRECTORY\_UNARCHIVER}
\lstinline;DIRECTORY_UNARCHIVER; accepts all headers that have the directory
typeflag ('5'). It will create a new directory and try to set the metadata
according to the header.

\section{Implementing a Custom UNARCHIVER}
To implement a custom \lstinline;UNARCHIVER;, one has to implement the following
features. Additional examples of custom \lstinline;UNARCHIVER;s can be found in
the examples directory (minipax's \lstinline;HEADER_SAVE_UNARCHIVER; and
tar\_ls' \lstinline;HEADER_PRINT_UNARCHIVER;).

\subsection{Creation Procedures}
All creation procedures have to call \lstinline;default_create; either using
\lstinline;Precursor; or by calling it directly.

\subsection{required\_blocks}
\lstinline;required_blocks: INTEGER;\\
Has to return the number of blocks that are required to unarchive the payload
that belongs to this header.

\subsection{can\_unarchive}
\lstinline;can_unarchive (a_header: TAR_HEADER): BOOLEAN;\\
Indicates whether \lstinline;a_header; can be unarchived by this
\lstinline;UNARCHIVER; type. This is the function, \lstinline;ARCHIVABLE; uses
to determine which \lstinline;UNARCHIVER; it should chose.

\subsection{unarchive\_block}
\lstinline|unarchive_block (a_block: MANAGED_POINTER; a_pos: INTEGER)| \\
Takes a block and unarchives its contents. The block starts at
\lstinline;a_pos;. This feature unarchives at most
\lstinline;{TAR_CONST}.tar_block_size; blocks. It has to increase
\lstinline;unarchived_blocks; by one.

\subsection{do\_internal\_initialization}
Initialize internal structures (only the ones this \lstinline;UNARCHIVER;
introduces). It is called automatically by \lstinline;initialize;.

\subsection{Utility Features}
To implement the features listed above, the following utility features are
provided.

\subsubsection{Bytes to Blocks}
The feature \lstinline|needed_blocks (n: NATURAL_64): NATURAL_64)| can be used
to determine how many blocks are required to store \lstinline;n; bytes.

\subsection{UID to Username}
To convert a uid to a username, \lstinline|file_owner (uid: INTEGER): STRING|
can be used.

\subsection{GID to Groupname}
To convert a gid to a groupname, \lstinline|file_group (gid: INTEGER): STRING|
can be used.
