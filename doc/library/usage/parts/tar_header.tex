\chapter{TAR\_HEADER}
A \lstinline|TAR_HEADER| instance contains all metadata, tarfiles store. The
parts about size, mtime and typeflag are based on \cite[Shell \& Utilities - pax]{POSIX:2008}.

\section{Metadata}
\subsection{filename}
path to the file

\subsection{mode}
Traditional UNIX style mode (0777, 0644, \dots).

\subsection{user\_id}
User ID of the file owner.

\subsection{group\_id}
Group ID of the file group.

\subsection{size}
For files this contains the size in bytes. In case the header does not belong
to a file, this is either zero or unspecified by the posix standard, so leaving
it at its default value (0) is a good choice.

The only exception are directories, for which a non-zero size indicates the
maximal number of bytes that this directory is able to hold (if supported by
the OS). If the size is zero there is no such limit (or no OS support).

\subsection{mtime}
Modification time of the file at archiving time, measured in unix time (seconds
since 00:00:00 UTC on 1 January 1970).

\subsection{typeflag}
Indicates what payload type this header follows. The following values are
standardized:
\begin{itemize}
	\item['0'] Regular files ('\%U' is allowed for backward compatibilty but
		should not be used)
	\item['1'] Hardlink (only allowed if the content was archived in an earlier
		entry)
	\item['2'] Symlink
	\item['3'] Character special device
	\item['4'] Block special device
	\item['5'] Directory
	\item['6'] FIFO
	\item['7'] Reserved for files to which an implementation has associated some
		high-performance attribute. May treat it as regular file.
	\item['A'-'Z'] Reserved for custom implementations
	\item Everything else is reserved for future standardization.
\end{itemize}

\subsection{linkname}
Target (pointee) of a link-type entry.

\subsection{user\_name}
Username of the file owner.

\subsection{group\_name}
Groupname of the file group.

\subsection{device\_major}
Device major number of a character or block device.

\subsection{device\_minor}
Device minor number of a character or block device.
